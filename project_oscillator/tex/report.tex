\documentclass[12pt,a4paper]{article}

% ===== Pacchetti principali =====
\usepackage[utf8]{inputenc}
\usepackage[T1]{fontenc}
\usepackage{lmodern}
\usepackage{geometry}
\geometry{a4paper, margin=2.5cm}
\usepackage{graphicx}
\usepackage{mathtools, amssymb}
\usepackage{caption}
\usepackage{subcaption}
\usepackage{hyperref}
\usepackage{float}
\usepackage{xcolor}
\usepackage{microtype}
\usepackage{cleveref}
\usepackage{listings}

% ===== Impostazioni listings per MATLAB =====
\lstset{
    language=Matlab,
    basicstyle=\ttfamily\small,
    keywordstyle=\color{blue},
    commentstyle=\color{gray},
    stringstyle=\color{red},
    breaklines=true,
    frame=single,
    numbers=left,
    numberstyle=\tiny
}

% ===== Titolo e autore =====
\title{Optimal Control via CPSO: From Theory to Implementation}
\author{Giovanni Giuseppe Iacuzzo}
\date{\today}

\begin{document}

\maketitle

\begin{abstract}
This report investigates the optimal control of a second-order dynamical system using Continuous Particle Swarm Optimization (CPSO). The study covers theoretical formulation, numerical integration techniques, implementation in MATLAB, and analysis of results.
\end{abstract}

\tableofcontents
\newpage

% ===== 1. Introduction =====
\section{Introduction}
The aim of this work is to recover a target control signal that drives a second-order system to a desired trajectory. The study combines theoretical background, numerical simulations, and population-based optimization methods.

The report is structured as follows:
\begin{itemize}
    \item \textit{Theoretical background:} formulation of the control problem.
    \item \textit{Forward problem:} numerical methods for ODE integration.
    \item \textit{CPSO implementation:} parameterization and optimization procedure.
    \item \textit{Results:} post-processing, visualization, and analysis.
\end{itemize}

% ===== 2. Theoretical Background =====
\section{Theoretical Background}
We consider a second-order system governed by the ODE:
\begin{equation}
\label{eq:system}
    u''(t) + \alpha u'(t) + \beta u(t) = g(t), \quad u(0) = u_0, \quad u'(0) = v_0,
\end{equation}
where $u(t)$ is the state, $g(t)$ the control input, and $\alpha, \beta$ are system parameters.

The optimal control $g(t)$ minimizes the cost functional:
\begin{equation}
\label{eq:cost}
    J(g) = \int_0^T \big(u(t) - u_{\mathrm{target}}(t)\big)^2 \, dt + \lambda \sum_{i=1}^{M} p_i^2,
\end{equation}
where $u_{\mathrm{target}}(t)$ is the desired trajectory, $p_i$ the control parameters, and $\lambda$ a regularization weight.

% ===== 3. Forward Problem =====
The system is discretized using a second-order finite difference scheme:
\begin{align}
    u_{n+1} &= \frac{1}{A} \Big( g_n + c_1 u_n - c_2 u_{n-1} - c_3 u_{n-1} - \beta u_n \Big), \\
    A &= \frac{1}{\Delta t^2} + \frac{\alpha}{2 \Delta t}, \quad
    c_1 = \frac{2}{\Delta t^2}, \quad
    c_2 = \frac{1}{\Delta t^2}, \quad
    c_3 = \frac{\alpha}{2 \Delta t}.
\end{align}

Initial values:
\begin{align}
    u_0 &= u(0), \\
    u_1 &= u_0 + \Delta t \, v_0 + \frac{\Delta t^2}{2} \big(g(0) - \alpha v_0 - \beta u_0 \big).
\end{align}

% ===== 4. Control Parameterization =====
Control signals are parameterized using:
\begin{itemize}
    \item \textit{Piecewise constant segments.}
    \item \textit{Cubic spline interpolation} of control parameters $p_i$.
\end{itemize}
Parameters are bounded: $p_{\min} \le p_i \le p_{\max}$.

% ===== 5. Implementation =====
\section{Implementation}
Implemented in MATLAB with key functions:
\begin{lstlisting}
forward_solver.m       % numerical ODE integration
build_control.m        % constructs g(t) from parameters
objective_cpso.m       % evaluates cost function
postprocess.m          % results analysis and visualization
\end{lstlisting}

Workflow:
\begin{enumerate}
    \item Load system parameters and target trajectory.
    \item Initialize CPSO parameters and run optimization.
    \item Compute system response using optimal control.
    \item Generate plots and statistics.
\end{enumerate}

\end{document}
